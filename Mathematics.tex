%% Mathematics

\newcommand{\Runderline}[1]{%
\begin{tikzpicture}[anchor=base,baseline]
    \node[inner sep=0,outer sep=0] (N) at (0,0) {#1};
    \coordinate (L) at ($(N.base west)+(0.1ex,-0.3ex)$);
    \coordinate (R) at ($(N.base east)+(-0.1ex,-0.3ex)$);
    \draw (L) -- (R);
\end{tikzpicture}
}

\newcommand{\Rdoubleunderline}[1]{%
\begin{tikzpicture}[anchor=base,baseline]
    \node[inner sep=0,outer sep=0] (N) at (0,0) {#1};
    \coordinate (L) at ($(N.base west)+(0.1ex,-0.3ex)$);
    \coordinate (R) at ($(N.base east)+(-0.1ex,-0.3ex)$);
    \draw (L) -- (R);
    \draw ($(L)+(0,-0.3ex)$) -- ($(R)+(0,-0.3ex)$);
\end{tikzpicture}
}

\newcommand{\Runderarrow}[1]{%
\begin{tikzpicture}[anchor=base,baseline]
    \node[inner sep=0,outer sep=0] (N) at (0,0) {#1};
    \coordinate (L) at ($(N.base west)+(0.1ex,-0.3ex)$);
    \coordinate (R) at ($(N.base east)+(-0.1ex,-0.3ex)$);
    \draw (L) -- (R) -- ++(-0.3ex,-0.3ex);
\end{tikzpicture}
}

\newcommand{\Rvec}[1]{\Runderline{\ensuremath{#1}}}
\newcommand{\Rmat}[1]{\Rdoubleunderline{\ensuremath{#1}}}
\newcommand{\Rdir}[1]{\Runderarrow{\ensuremath{#1}}}

%% End
