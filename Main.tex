\documentclass{book}
%% Packages

% Font
\usepackage{newpxtext}
\usepackage[euler-digits,euler-hat-accent]{eulervm}

% Page Config
\usepackage{geometry}
\geometry{%
    a4paper,
    top=1in,
    bottom=1in,
    left=1in,
    right=1in,
    %bindingoffset=5mm
}
\usepackage{fancyhdr}

% Mathematics
\usepackage{amsmath}
\usepackage{amssymb}
\usepackage{amstext}
\usepackage{mathtools}
\usepackage{array}

%% Code
\usepackage[cache=false]{minted}
\usepackage{listings}

% Graphics
\usepackage{graphicx}
\usepackage{tikz}
    \usetikzlibrary{arrows}
    \usetikzlibrary{arrows.meta}
    \usetikzlibrary{patterns}
    \usetikzlibrary{shapes}
    \usetikzlibrary{calc}
    \usetikzlibrary{matrix}
\usepackage{pgfplots}
\usepackage{xcolor}

% Tables of Contents
%\usepackage{tocloft}
%\usepackage{morewrites}

% Hyperlinks
\usepackage{hyperref}

%% End

%% Lengths

\setlength{\parskip}{1em}
\setlength{\columnsep}{3ex}
\newcommand\K{\kern.05ex}
%\linespread{2}

%% End

%% Error tools

\newcounter{UNDEFerr}
\newcommand{\UNDEF}{%
    \stepcounter{UNDEFerr}
    \PackageWarning{UNDEF}{Use of undefined command (\arabic{UNDEFerr})}
    \underline{\textbf{UNDEF \#\arabic{UNDEFerr}}}
}
\newcommand{\HIDE}[1]{%
    \PackageWarning{HIDE}{Hidden text, page \arabic{page},}
}
\newcommand{\TODO}[1]{%
    \texttt{TODO}: \textit{#1}\PackageWarning{TODO}{"#1", page \arabic{page},}
}

%% End

% Header settings

\newcounter{Rsubpage}
\newcommand{\Rcalcsubpage}{%
    \setcounter{Rsubpage}{\value{page}-\value{Rchapterstart}}
}

\fancypagestyle{Rmain}{%
    \fancyhf{}
    \renewcommand{\headrulewidth}{1pt}
    \renewcommand{\footrulewidth}{1pt}
    \setlength{\headheight}{14pt}
    %
    \lhead{\Roman{Rvolume}:\arabic{Rchapter}}
    \chead{\textbf{\Rcurrentchapter}}
    \rhead{\Rcalcsubpage\arabic{Rsubpage}}
    %
    \cfoot{\arabic{page}}
}
\pagestyle{Rmain}

%% End

%% Page tools

\newcommand{\blankpage}{%
    \thispagestyle{empty}
    ~\newpage
}

\newcommand{\newpagerecto}{%
    \newpage
    \ifodd\value{page}
    \else
        \blankpage
    \fi
}

\newcommand{\newpageverso}{%
    \newpage
    \ifodd\value{page}
        \blankpage
    \else
    \fi
}

%% End

%% Sectioning

\newcounter{Rvolumestart}
\newcounter{Rchapterstart}

\newcounter{Rvolume}
\newcounter{Rchapter}
\newcounter{Rsection}
\newcounter{Rsubsect}
\newcommand{\Rcurrentvolume}{\UNDEF}
\newcommand{\Rcurrentchapter}{\UNDEF}
\newcommand{\Rcurrentsection}{\UNDEF}
\newcommand{\Rcurrentsubsect}{\UNDEF}
\renewcommand{\theRvolume}{\Roman{Rvolume}}
\renewcommand{\theRchapter}{\arabic{Rchapter}}

% toc file streams
\newwrite\Rmajortocfile
\newwrite\Rminortocfile
\immediate\openout\Rmajortocfile=RSkylInternal/MajorContents.new
\immediate\write\Rmajortocfile{%
    \noexpand\newcommand{\noexpand\Rmajortocincluded}{true}
}
%\immediate\closeout\Rmajortocfile

\newcommand{\Rtitlepage}[4]{%
    \thispagestyle{empty}
  \vspace*{\fill}
    ~
  \vspace*{\fill}
    \begin{center}
        {\large\textbf{#1}}\\
        \rule[1ex]{\textwidth}{1pt}\\
        {\Huge\textbf{#2}}\\
        \rule[1ex]{\textwidth}{1pt}\\
        {\large\textbf{#3}}
    \end{center}
  \vspace*{\fill}
        #4
  \vspace*{\fill}
    ~
  \vspace*{\fill}
    ~
  \vspace*{\fill}
    \newpage
}

\newcommand{\Rmajortoc}{%
    \noindent\textbf{Table of Contents}\\[1ex]
    \rule[1ex]{\textwidth}{1pt}\\[1em]
    \newcommand {\Rmajortocincluded }{true} 
\Rtocline {1cm}{I.}{\textbf {Page Layout}}{2}{RSkyl:V:1:0:0:0} 
\Rtocline {1cm}{II.}{\textbf {Testing}}{4}{RSkyl:V:2:0:0:0} 
\Rtocline {1cm}{III.}{\textbf {Test Volume}}{11}{RSkyl:V:3:0:0:0} 
\Rtocline {1.5cm}{1.}{\textbf {Test Chapter}}{11}{RSkyl:C:3:1:0:0} 
\Rtocline {2cm}{1.}{Test Section}{11}{RSkyl:S:3:1:1:0}
\Rtocline {2.5cm}{i.}{Test Subsect}{11}{RSkyl:SS:3:1:1:1}

    \lstinputlisting{RSkylInternal/MajorContents.tex}
    \ifdefined\Rmajortocincluded
    \else
        \TODO{Major ToC (rebuild required)}
    \fi
}

\newcommand{\Rtocline}[4]{%
    \noindent
    \parbox[t]{#1}{%
        \hfill
        #2
        \hspace*{2pt}
    }
    \textbf{#3}
    \phantom{\rule[-0.5em]{0pt}{0.5em}}
    \hfill
    #4
    \hspace*{1cm}\\
}

\newcommand{\Rvolume}[1]{%
    \refstepcounter{Rvolume}
    \setcounter{Rvolumestart}{\value{page}}
    \renewcommand{\Rcurrentvolume}{#1}
    \immediate\write\Rmajortocfile{%
        \noexpand\Rtocline{1cm}{\Roman{Rvolume}.}{#1}{\arabic{page}}
    }
    %
    \setcounter{Rchapter}{0}
    \setcounter{Rsection}{0}
    \setcounter{Rsubsect}{0}
    %
    \thispagestyle{empty}
    ~\vspace{1in}
    \begin{center}
        {\large\textbf{Volume \Roman{Rvolume}}}\\
        \rule[1ex]{0.5\textwidth}{1pt}\\
        {\huge\textbf{#1}}
    \end{center}
}

%% End

%% Enumerate / Itemize

\newcommand{\Rbullet}[1][]{%
    \begin{tikzpicture}[baseline=0pt,#1]
        \fill (0pt,0.5ex) circle (1pt);
    \end{tikzpicture}%
}

\setlist{topsep=0pt}
\setlist[itemize]{label=\Rbullet}
\setlist[enumerate,1]{label=\arabic*.,ref=\arabic*.}
\setlist[enumerate,2]{label=\alph*.,ref=\theenumi\alph*.}
\setlist[enumerate,3]{label=\roman*.,ref=\theenumii\roman*.}

%% End


%\usepackage{lua-visual-debug}

%\usepackage[generate,ps2eps]{abc}

\newcommand{\Rsamplefox}{Giffib:\ ``The\ quick\ brown\ fox\ jumped\ over\ the\ lazy\ dog.''}
\newcommand{\Rsampleupp}{ABCDEFGHIJKLMNOPQRSTUVWXYZ}
\newcommand{\Rsamplelow}{abcdefghijklmnopqrstuvwxyz}
\newcommand{\Rsamplenum}{0123456789}
\newcommand{\Rsampleall}{\Rsamplefox\\\Rsampleupp\\\Rsamplelow\\\Rsamplenum}

%\newwrite\tempfile

%% TESTING GROUND

\begin{document}

\Rtitlepage{Alexander J. Johnson}{R's Book}{Ranford Skyline's Book Template}{%
    Fully customised document template including:
    \begin{itemize}
        \item Custom document hierarchy: book, part, chapter, section,
            subsection.
        \item Custom tables of contents.
        \item Custom fonts.
        \item Support for:
            \begin{itemize}
                \item Linguistics,
                \item Hyperlinks,
            \end{itemize}
        \item Some other stuff.
    \end{itemize}
}

\Rmajortoc

\Rvolume{Page Layout}

\newpage
\layout

\Rvolume{Testing}
\renewcommand{\Rcurrentchapter}{The Testing Ground}

\textbf{Standard} - \Rsampleall

\textbf{Italics} - \textit{\Rsampleall}

\textbf{Bold} - \textbf{\Rsampleall}

\textbf{Bold Italics} - \textbf{\textit{\Rsampleall}}

\textbf{Typewriter} - \texttt{\Rsampleall}

\textbf{Small Caps} - \textsc{\Rsampleall}

\textbf{Math Mode} - $\Rsampleall$

\textbf{Math Calligraphic} - $\mathcal{\Rsampleupp}$

\textbf{Math Backboard} - $\mathbb{\Rsampleupp}$

\textbf{Math Fraktur} - $\mathfrak{\Rsamplefox}$\\$\mathfrak{\Rsampleupp}$\\%
$\mathfrak{\Rsamplelow}$\\$\mathfrak{\Rsamplenum}$


\newpage

%\textit{wund\textbf{\underline{\u{V}}}de}

\lipsum[1]

\newcounter{alpha}
\newcounter{gamma}
\setcounter{gamma}{1}
\ifnum\value{alpha}=\value{gamma}
    True
\else
    False
\fi

Hello World!
\begin{align*}
    %y_j &= \xi^j
    B_n(t) &= \sum_{i=0}^{n} \begin{pmatrix} n \\ i \end{pmatrix} t^i
        \left(\sum_{j=0}^{i} \begin{pmatrix} i \\ j \end{pmatrix} x_j(-1)^{i-j}\right)
\end{align*}

\TODO{This bit}
\HIDE{This other bit}

\newpagerecto
Hi, Beans!
\newpageverso
Go away\dots

\newpage
This is the start of some text:
\begin{enumerate}
    \item enumeration item number one;
    \item another item,
        \begin{enumerate}
            \item nested enumerate,
            \item more stuff;
        \end{enumerate}
    \item ``A can o' beans.''
\end{enumerate}
More text that follows after it:
\begin{itemize}
    \item itemize item number one;
    \item another item,
        \begin{itemize}
            \item nested itemize,
            \item more stuff;
        \end{itemize}
    \item ``No more beans for you...''
\end{itemize}
\lipsum[2-3]
\begin{enumerate}
    \item alpha \label{alpha}
        \begin{enumerate}
            \item beta \label{beta}
                \begin{enumerate}
                    \item gamma \label{gamma}
                \end{enumerate}
        \end{enumerate}
\end{enumerate}
alpha \ref{alpha},
beta \ref{beta},
gamma \ref{gamma}.

\noindent Alla nasa ero monite\\
Alla nasa ero monite\\
polo Anne jena golly quack\\
polo Anne jena golly quack\\

A test of the IPA command \ipa{/ɹʌka/}.
\begin{center}
    \forestapplylibrarydefaults{linguistics}
    \begin{forest}
        syntax
        [S  [I  [Q  [V [Do]]
                    [N [you]]
                ]
                [VP [AV [really]]
                    [V [like]]
                ]
            ]
            [Det [N [my]]
                 [NP [Adj [red]]
                     [N [dress ?, roof]]
                 ]
            ]
        ]
    \end{forest}
\end{center}

\Rmusic{lilypondTest}

\begin{align*}
    0 &= \partial_\mu\left|\Rvec{l}(\mu)-\Rvec{p}\right|\\
    \left|\Rvec{l}-\Rvec{p}\right|^2 &= \left|\Rvec{a} + \mu\Rdir{ab} - \Rvec{p}\right|^2\\
\end{align*}

Quaternions are defined by the equation
\begin{align*}
    \imath^2 = \jmath^2 = \kappa^2 = \imath\jmath\kappa = -1.
\end{align*}
By considering pairs of terms, it is found that:
\begin{align*}
    \imath\jmath\kappa &= \imath^2, & \jmath\kappa &= \imath,\\
    \imath^2 = \jmath\kappa\imath &= \jmath^2, & \kappa\imath &= \jmath,\\
    \imath\jmath\kappa &= \kappa^2, & \imath\jmath &= \kappa.
\end{align*}
By multiplying pairs of terms, it is found that:
\begin{align*}
    \imath\jmath^2 = \imath\jmath\jmath = \kappa\jmath = -\imath,\\
    \imath^2\jmath = \imath\imath\jmath = \imath\kappa = -\jmath,\\
    \jmath^2\kappa = \jmath\jmath\kappa = \jmath\imath = -\kappa.
\end{align*}
Note that the multiplication of quaternions is associative, but not commutative.

Ki\ng\ of \th e dirt.

\newpage
Code test
\inputminted{lua}{MakeDoc.lua}

\newpage
\Rvolume{Test Volume}
%
Some volume text

\Rminortoc

\Rchapter{Test Chapter}
%
Some chapter text

\Rsection{Test Section}
%
Some section text

\Rsubsect{Test Subsect}
%
Some subsection text

\end{document}
